% vim: set ft=tex spell
\section{Provisioning}
\label{sec:provisioning}

This section will describe how to provision a fully functional Consus inventory management system.
Consus is comprised of two parts: a server and a client.

Whilst it is possible to run a Consus server and client on ``Computer A'' and only a Consus client on ``Computer B'', we do not recommend it. % TODO why?
If you want to use multiple check-out stations, we recommend that you have a dedicated Consus server; that is, do not run a Consus server and Consus client on the same computer.

Otherwise, if you are only using one check-out station, you can run the Consus server and client on the same computer.

\subsection{Server}
\label{subsec:server_provision}

The Consus server is where all the magic happens; it performs all the business logic for Consus.
The server is required so that it is possible to have multiple client check-out stations working concurrently with the same data.
For instructions on provisioning a Consus client, see~\autoref{subsec:client_provision}.

\subsubsection{Requirements}

\blindtext%

\subsection{Client}
\label{subsec:client_provision}

The Consus client is where you as the user get a pretty\footnote{Beauty is subjective, okay?} view of the data on the Consus server.
Since the client doesn't keep any data, a server is also required (see~\autoref{subsec:server_provision}).

\subsubsection{Requirements}

\blindtext%
